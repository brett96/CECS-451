\documentclass{article}

\usepackage{amsmath}
\usepackage{algorithmic}
\usepackage{graphicx}
\usepackage{xspace}
\usepackage{multirow}

\begin{document}
\title{Hello \LaTeX \xspace World}
\date{}

\author {Brett \\Tomita \\
\texttt{bretttomita@gmail.com}}
\maketitle

\begin{abstract}
``This document is a model and instructions for \LaTeX\xspace\xspace world''
\end{abstract}

\section{Introduction}
Welcome to the \LaTeX\xspace world.

\section{Ease of Use}

\subsection{Maintaining the Integrity of the Specifications}
The `article' class is used to format your paper and style the text. All margins, column widths, line spaces, and text fonts are prescribed.

\section{Styling Guide}

\subsection{Abbreviations and Acronyms}
Define abbreviations and acronyms the first time they are used in the text, 
even after they have been defined in the abstract.

\subsection{Equations}
\begin{equation}
%1+1=2
\sum_{n=0}^{\infty} \frac{af^n}{n!}(x-a)^n
\label{Taylor}
\end{equation}
(1) is the famous Taylor series. Use ``(1)'', not ``Eq. (1)'' or equation (1)'',
except at the beginning of a sentence: ``Equation (1) is . . .

Taylor series in a text would be
$\sum_{n=0}^{\infty} \frac{af^n}{n!}(x-a)^n$

\subsection{Lists}
Bullet style list.
\begin{itemize}
    \item I am one
    \item I am two
    \item I am three
\end{itemize}

Number style list.
\begin{enumerate}
    \item I am one
    \item I am two
    \item I am three
\end{enumerate}

\subsection{Figures and Tables}
\paragraph{Positioning Figures and Tables} Figure captions should be below the figures; table heads should appear above the tables. Insert figures and tables after they are cited in the text. Use the abbreviation .

\begin{table}[h]

\caption{Table Type Styles}
\begin{center}
\begin{tabular}{|c||l|l|l|}
  \hline
  \multirow{2}{*}{\textbf{Table Head}} 
      & \multicolumn{3}{c|}{\textbf{Table Column Head}} \\  \cline{2-4}
  & \textbf{\emph{Table column subhead}} & \textbf{\emph{Subhead}} & \textbf{\emph{Subhead}}  \\  \hline
  $$ &  &  &  \\      \hline
\end{tabular}
\label{myTable}
\end{center}
\end{table}
%\ref{myTable}

\begin{figure}[h]
\centering
\includegraphics[width=0.4\columnwidth]{fig1.png}
\\\title{Figure 1: Working Example}
\end{figure}

\subsection{Algorithms}
\begin{algorithmic}
 \STATE $i \xleftarrow[]{}10$
 \IF {i $\geq$ 5}
 \STATE $i \xleftarrow[]{}i-1$
 \ELSE
 \IF{ i $\leq$ 3}
 \STATE $i \xleftarrow[]{} i + 2$
 \ENDIF
 \ENDIF
 
\end{algorithmic}

\subsection{Source codes}
\begin{verbatim}
public class HelloWorld
{
    public static void main(String[] args 
    {
        System.out.println("Hello")
    }
}
\end{verbatim}

\subsection{References}
The first reference is \cite{Eason}, the second one is \cite{Maxwell}, and the
last one is \cite{Jacobs}

\begin{thebibliography}{}
\bibitem{Eason} G. Eason, B. Noble, and I. N. Sneddon, ``On certain integrals of Lipschitz-Hankel type involving products of Bessel functions,'' Phil. Trans. Roy. Soc. London, vol. A247, pp. 529--551, April 1955.
\bibitem{Maxwell} J. Clerk Maxwell, A Treatise on Electricity and Magnetism, 3rd ed., vol. 2. Oxford: Clarendon, 1892, pp.68--73.
\bibitem{Jacobs} I. S. Jacobs and C. P. Bean, ``Fine particles, thin films and exchange anisotropy,'' in Magnetism, vol. III, G. T. Rado and H. Suhl, Eds. New York: Academic, 1963, pp. 271--350
\end{thebibliography}

\end{document}
